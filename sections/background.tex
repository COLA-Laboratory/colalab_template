%!TeX root=../main.tex
% -----------------------------------------------------------------------------
% sections/background.tex
% Purpose: Background, notation, definitions, and related work.
%
% Keep this section concise: define only what is needed to understand the rest
% of the report. Move long derivations or extended proofs to the appendix.
% -----------------------------------------------------------------------------

\section{Background and Preliminaries}\label{sec:background}


\subsection{Notation (Example Table)}\label{sec:notation}

\pref{tab:notation} shows an example notation table. Replace the symbols and descriptions with those relevant to your report.

\begin{table}[t!]
    \centering
    \caption{Example notation table.}\label{tab:notation}
    \begin{threeparttable}
        \begin{tabular}{ll}
            \toprule
            Symbol & Meaning \\
            \midrule
            $n$ & Number of samples \\
            $d$ & Feature dimension \\
            $x_i \in \mathbb{R}^d$ & Input feature vector \\
            $y_i$ & Target / label \\
            $\theta$ & Model parameters \\
            \bottomrule
        \end{tabular}
    \end{threeparttable}
\end{table}

\subsection{Definitions and Theorems (Examples)}\label{sec:definitions}

\begin{definition}[Example Definition]\label{def:technical_report}
A \emph{technical report} is a document that describes a technical problem, the adopted approach, and supporting evidence (e.g., experiments, proofs, or analyses) in a reproducible way.
\end{definition}

We will refer back to \pref{def:technical_report} to illustrate cross-referencing in the template.

\begin{theorem}[Example Theorem]\label{theorem:amgm}
For any real numbers $a$ and $b$, we have $a^2 + b^2 \ge 2ab$.
\end{theorem}

\begin{proof}
The claim follows from the fact that $(a-b)^2 \ge 0$, which expands to $a^2 + b^2 - 2ab \ge 0$.
\end{proof}

\subsection{Example Equation}\label{sec:example_equation}

Use numbered equations for results that you will reference. For example, a standard regularized least-squares objective is:
\begin{equation}
\min_{\theta}\quad
\mathcal{L}(\theta)
=
\frac{1}{n}\sum_{i=1}^{n}\big\|f_{\theta}(x_i) - y_i\big\|_2^2
\;+\;
\lambda\|\theta\|_2^2,
\label{eq:least_squares}
\end{equation}
where $\lambda \ge 0$ controls the amount of regularization.

\subsection{Related Work (Placeholder)}\label{sec:related_work}
