%!TeX root=../main.tex
% -----------------------------------------------------------------------------
% sections/methods.tex
% Purpose: Describe the methodology used in the report.
%
% Recommended contents:
% - High-level overview / design rationale
% - Formal problem definition (if not already in background)
% - Core algorithm(s) with pseudo-code
% - Complexity / resource analysis
% - Implementation details that affect reproducibility
% -----------------------------------------------------------------------------

\section{Methods}
\label{sec:methods}


\subsection{Approach Overview}
\label{sec:approach_overview}

At a high level, the method takes input data $D$, applies a sequence of processing steps, and produces an output artifact (e.g., a model, a set of results, or a system). Refer back to \pref{fig:system_overview} for the placeholder overview diagram.

\subsection{Core Objective (Example Equation)}
\label{sec:objective}

If your method can be described as an optimization problem, write the objective explicitly and reference it later. For example:
\begin{equation}
\min_{\theta}\quad
\mathcal{J}(\theta) = \mathcal{L}(\theta) + \alpha \mathcal{R}(\theta),
\label{eq:generic_objective}
\end{equation}
where $\mathcal{L}$ is a data-fit term, $\mathcal{R}$ is a regularizer, and $\alpha \ge 0$ balances the two.

\subsection{Algorithm (Example Pseudocode)}
\label{sec:algorithm}

\pref{alg:training_loop} shows a minimal example using \texttt{algorithm2e}. Replace it with your actual algorithm.

\begin{algorithm}[t!]
    \DontPrintSemicolon
    \KwIn{Dataset $D=\{(x_i,y_i)\}_{i=1}^{n}$, hyperparameters $\eta$}
    \KwOut{Trained parameters $\theta$}

    Initialize $\theta$ (e.g., randomly)\;
    \For{$t \gets 1$ \KwTo $T$}{
        Sample a minibatch $B \subseteq D$\;
        Compute gradient $g \gets \nabla_{\theta}\mathcal{J}(\theta; B)$\;
        Update $\theta \gets \theta - \eta g$\;
    }
    \Return{$\theta$}\;
    \caption{Example training loop.}
    \label{alg:training_loop}
\end{algorithm}

\subsection{Complexity and Resources (Placeholder)}
\label{sec:complexity}
